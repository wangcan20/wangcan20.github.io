\documentclass{resume} % Use the custom resume.cls style

\usepackage[left=0.4 in,top=0.4in,right=0.4 in,bottom=0.4in]{geometry} % Document margins
\newcommand{\tab}[1]{\hspace{.2667\textwidth}\rlap{#1}} 
\newcommand{\itab}[1]{\hspace{0em}\rlap{#1}}
\name{Can Wang} % Your name
\address{
Johns Hopkins University \\
Biostatistics \\
\href{mailto:cwang271@jh.edu}{cwang271@jh.edu} \\
\href{https://github.com/wangcan20}{github.com/wangcan20}
}


\begin{document}

%----------------------------------------------------------------------------------------
%	EDUCATION SECTION
%----------------------------------------------------------------------------------------

\begin{rSection}{Research Interests}

\begin{itemize}
    \itemsep -3pt {}
    \item Distribution-free inference; Uncertainty quantification
    \item AI for Statistics; Medical AI
    \item Survival analysis; Risk Prediction
    \item Causal inference
    
    
\end{itemize}



\end{rSection}

\begin{rSection}{Education}

{\bf Johns Hopkins University} \\
Master of Biostatistics $\mid$ GPA: 4.00/4.00 \hfill {Aug 2024 - May 2026 (expected)} \\
\textit{Selected Coursework:} Probability and Statistical Inference (I–IV); Methods in Biostatistics (I–IV); Real Analysis; Causal Inference; Survival Analysis; Statistical Computing; Data Management.\\
\textbf{Thesis:} A Two-stage Conformal Predictive Interval for Competing-Risk Survival Data\\
\textit{Passed Departmental Comprehensive Exam, Jun 2025.}

{\bf Tsinghua University} \\
Bachelor of Industrial Engineering $\mid$ GPA: 3.53/4.00 \hfill {Sep 2020 - Jul 2024} \\ 
\textit{Selected Coursework:}
\begin{itemize}\itemsep -3pt
    \item \textbf{Math:} Calculus (A,B); Linear Algebra; Probability; Operations Research I–II (Deterministic \& Stochastic).
    \item \textbf{Statistics \& Data Science:} Statistics and Data Analytics; Linear Regression; Multivariate Statistics; Experimental Design; Statistical Computing and Software; Machine Learning and Big Data; Modeling and Simulation; Biostatistics; Causal Inference.
    \item \textbf{Computer Science:} Programming (C++, Python); Data Structures and Algorithms; Database Systems.
\end{itemize}
\textbf{Thesis:} Graph-Based Optimization for Assembly Sequence Planning under Uncertainty
\end{rSection}



%----------------------------------------------------------------------------------------
%	PUBLICATION SECTION
%----------------------------------------------------------------------------------------
\begin{rSection}{PUBLICATION/PREPRINTS}
\vspace{-1.25em}
\item \textbf{Wang, C.}, \& Chen, Y. (2025). Evaluating Large Language Models for Evidence-Based Clinical Question Answering (No. arXiv:2509.10843). \textit{arXiv}. \url{https://doi.org/10.48550/arXiv.2509.10843}
\end{rSection}

%----------------------------------------------------------------------------------------
%	RESEARCH SECTION
%----------------------------------------------------------------------------------------
\begin{rSection}{RESEARCH EXPERIENCE}

\textbf{Two-stage Conformal Predictive Interval for Competing-Risk Survival Data} \hfill Aug 2025 -- Present\\
Johns Hopkins University $\mid$ Advisor: Prof.~Mei-Cheng Wang \hfill \textit{Baltimore, MD}
\begin{itemize}\itemsep -3pt
    \item Developed a two-stage conformal inference framework for competing-risk survival data, constructing conformal support for cause type prediction and conformal intervals for cause-specific time-to-event.
    \item Introduced a nonparametric joint estimator of event time, covariates, and cause type to generate synthetic calibration samples under censoring, enabling model-free calibration.
    \item Performed extensive simulation studies across heterogeneous hazard structures, censoring levels, and sensitivity settings.
    \item Implementing an open-source \texttt{R} package and preparing a first-author manuscript.
\end{itemize}



\textbf{AI Agent for Causal Inference} \hfill Aug 2025 -- Present\\
Johns Hopkins University $\mid$ Advisor: Prof.~Yiqun Chen \hfill \textit{Baltimore, MD}
\begin{itemize}\itemsep -3pt
    \item Designed an autonomous causal estimation framework in which large language models generate, execute, and refine estimators with minimal human input.
    \item Evaluated state-of-the-art LLMs on the ACIC 2022 Track 2 benchmark; zero-shot programs produced canonical estimators and achieved mid-range RMSE–coverage performance relative to human submissions.
    \item Extended the system to a self-evolving coding agent using a MAP-Elites-style framework, where iterative updates incorporated meaningful refinements including regularization, covariate encoding, pre-mean and trend adjustments, and improved clustered inference.
    \item Across 100 iterations, the self-evolving agent matched or exceeded human-expert performance, achieving lower RMSE and near-nominal confidence interval coverage. First-authored Preprint available.

\end{itemize}




\textbf{AI for Evidence-Based Clinical QA (MEDAL Dataset)} \hfill Mar 2025 -- Sept 2025\\
Johns Hopkins University $\mid$ Advisor: Prof.~Yiqun Chen \hfill \textit{Baltimore, MD}
\begin{itemize}\itemsep -3pt
    \item Curated a multi-source benchmark (21.6K QA pairs) from Cochrane systematic reviews, AHA structured guidelines, and narrative guidelines to evaluate how LLMs answer clinical questions and reason over evidence quality.
    \item Analyzed multiple LLMs, showing that performance depends strongly on the structure, clarity and quality of source evidence. Model uncertainty increases when clinical evidence is weak or ambiguous.
    \item Demonstrated that retrieval-augmented prompting using PubMed abstracts substantially improves factual accuracy (from 60.3\% to 79.9\%), highlighting the role of external evidence in improving performance.
    \item Released the MEDAL dataset and evaluation framework as an open-source benchmark (\href{https://huggingface.co/datasets/cwang271/MEDAL}{HuggingFace}, \href{https://github.com/yiqunchen/MEDAL}{GitHub}); poster accepted at ML4H 2025; with first-authored preprint available.
\end{itemize}





\end{rSection}

%----------------------------------------------------------------------------------------
%	WORK EXPERIENCE SECTION
%----------------------------------------------------------------------------------------

\begin{rSection}{WORK EXPERIENCE}

\textbf{Bailongma Yunxing Technology Co., Ltd.} \hfill Jun 2023 -- Sep 2023 \\
\textit{Data Research and Development Department, Intern} \hfill \textit{Beijing, China}
\begin{itemize}\itemsep -3pt
    \item Extracted and cleaned 12.8M+ ride-hailing records using SQL and Python to study multi-homing behavior among drivers across major platforms.
    \item Conducted exploratory data analysis comparing behavioral and income patterns of multi-homing vs.\ single-homing drivers, producing visual summaries for internal reporting.
    \item Built a Random Forest model for feature selection and a Multinomial Logistic Regression model to predict multi-homing choice, achieving 79.5\% accuracy.
\end{itemize}



\end{rSection} 

\begin{rSection}{OTHER EXPERIENCE}

\begin{itemize}\itemsep -3pt
    \item \textbf{Teaching:} TA for AS.280.345 Public Health Biostatistics (FA25); led weekly discussion sessions and graded assignments.
    \item \textbf{Presentations:} Poster “Evaluating Large Language Models for Evidence-Based Clinical Question Answering” presented at JHU DSAI Symposium; to be presented at ML4H 2025.
    \item \textbf{Reviewing:} Reviewer for ML4H 2025.
\end{itemize}

\end{rSection}


%----------------------------------------------------------------------------------------
\begin{rSection}{SKILLS}

\textbf{Languages:} English (fluent; TOEFL 114); Chinese (native); Portuguese (beginner); Japanese (beginner).\\
\textbf{Programming:} Python, R, C++, SQL; Familiar with open-source development and reproducible pipelines (GitHub, R package development, Python modules).\\
\textbf{Personal Interests:} EDM production; multi-instrument performance; science fiction; strategic board games.

\end{rSection}


\end{document}
